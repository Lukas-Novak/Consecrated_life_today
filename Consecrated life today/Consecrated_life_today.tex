%  LaTeX support: latex@mdpi.com
%  In case you need support, please attach all files that are necessary for compiling as well as the log file, and specify the details of your LaTeX setup (which operating system and LaTeX version / tools you are using).

%=================================================================
\documentclass[ijerph,article,accept,moreauthors,pdftex]{mdpi}

% If you would like to post an early version of this manuscript as a preprint, you may use preprint as the journal and change 'submit' to 'accept'. The document class line would be, e.g., \documentclass[preprints,article,accept,moreauthors,pdftex]{mdpi}. This is especially recommended for submission to arXiv, where line numbers should be removed before posting. For preprints.org, the editorial staff will make this change immediately prior to posting.

%% Some pieces required from the pandoc template
\providecommand{\tightlist}{%
  \setlength{\itemsep}{0pt}\setlength{\parskip}{4pt}}
\setlist[itemize]{leftmargin=*,labelsep=5.8mm}
\setlist[enumerate]{leftmargin=*,labelsep=4.9mm}

\usepackage{longtable}

% see https://stackoverflow.com/a/47122900

%--------------------
% Class Options:
%--------------------
%----------
% journal
%----------
% Choose between the following MDPI journals:
% acoustics, actuators, addictions, admsci, aerospace, agriculture, agriengineering, agronomy, algorithms, animals, antibiotics, antibodies, antioxidants, applsci, arts, asc, asi, atmosphere, atoms, axioms, batteries, bdcc, behavsci , beverages, bioengineering, biology, biomedicines, biomimetics, biomolecules, biosensors, brainsci , buildings, cancers, carbon , catalysts, cells, ceramics, challenges, chemengineering, chemistry, chemosensors, children, cleantechnol, climate, clockssleep, cmd, coatings, colloids, computation, computers, condensedmatter, cosmetics, cryptography, crystals, dairy, data, dentistry, designs , diagnostics, diseases, diversity, drones, econometrics, economies, education, electrochem, electronics, energies, entropy, environments, epigenomes, est, fermentation, fibers, fire, fishes, fluids, foods, forecasting, forests, fractalfract, futureinternet, futurephys, galaxies, games, gastrointestdisord, gels, genealogy, genes, geohazards, geosciences, geriatrics, hazardousmatters, healthcare, heritage, highthroughput, horticulturae, humanities, hydrology, ijerph, ijfs, ijgi, ijms, ijns, ijtpp, informatics, information, infrastructures, inorganics, insects, instruments, inventions, iot, j, jcdd, jcm, jcp, jcs, jdb, jfb, jfmk, jimaging, jintelligence, jlpea, jmmp, jmse, jnt, jof, joitmc, jpm, jrfm, jsan, land, languages, laws, life, literature, logistics, lubricants, machines, magnetochemistry, make, marinedrugs, materials, mathematics, mca, medicina, medicines, medsci, membranes, metabolites, metals, microarrays, micromachines, microorganisms, minerals, modelling, molbank, molecules, mps, mti, nanomaterials, ncrna, neuroglia, nitrogen, notspecified, nutrients, ohbm, particles, pathogens, pharmaceuticals, pharmaceutics, pharmacy, philosophies, photonics, physics, plants, plasma, polymers, polysaccharides, preprints , proceedings, processes, proteomes, psych, publications, quantumrep, quaternary, qubs, reactions, recycling, religions, remotesensing, reports, resources, risks, robotics, safety, sci, scipharm, sensors, separations, sexes, signals, sinusitis, smartcities, sna, societies, socsci, soilsystems, sports, standards, stats, surfaces, surgeries, sustainability, symmetry, systems, technologies, test, toxics, toxins, tropicalmed, universe, urbansci, vaccines, vehicles, vetsci, vibration, viruses, vision, water, wem, wevj

%---------
% article
%---------
% The default type of manuscript is "article", but can be replaced by:
% abstract, addendum, article, benchmark, book, bookreview, briefreport, casereport, changes, comment, commentary, communication, conceptpaper, conferenceproceedings, correction, conferencereport, expressionofconcern, extendedabstract, meetingreport, creative, datadescriptor, discussion, editorial, essay, erratum, hypothesis, interestingimages, letter, meetingreport, newbookreceived, obituary, opinion, projectreport, reply, retraction, review, perspective, protocol, shortnote, supfile, technicalnote, viewpoint
% supfile = supplementary materials

%----------
% submit
%----------
% The class option "submit" will be changed to "accept" by the Editorial Office when the paper is accepted. This will only make changes to the frontpage (e.g., the logo of the journal will get visible), the headings, and the copyright information. Also, line numbering will be removed. Journal info and pagination for accepted papers will also be assigned by the Editorial Office.

%------------------
% moreauthors
%------------------
% If there is only one author the class option oneauthor should be used. Otherwise use the class option moreauthors.

%---------
% pdftex
%---------
% The option pdftex is for use with pdfLaTeX. If eps figures are used, remove the option pdftex and use LaTeX and dvi2pdf.

%=================================================================
\firstpage{1}
\makeatletter
\setcounter{page}{\@firstpage}
\makeatother
\pubvolume{xx}
\issuenum{1}
\articlenumber{5}
\pubyear{2019}
\copyrightyear{2019}
%\externaleditor{Academic Editor: name}
\history{Received: date; Accepted: date; Published: date}
\updates{yes} % If there is an update available, un-comment this line

%% MDPI internal command: uncomment if new journal that already uses continuous page numbers
%\continuouspages{yes}

%------------------------------------------------------------------
% The following line should be uncommented if the LaTeX file is uploaded to arXiv.org
%\pdfoutput=1

%=================================================================
% Add packages and commands here. The following packages are loaded in our class file: fontenc, calc, indentfirst, fancyhdr, graphicx, lastpage, ifthen, lineno, float, amsmath, setspace, enumitem, mathpazo, booktabs, titlesec, etoolbox, amsthm, hyphenat, natbib, hyperref, footmisc, geometry, caption, url, mdframed, tabto, soul, multirow, microtype, tikz

%=================================================================
%% Please use the following mathematics environments: Theorem, Lemma, Corollary, Proposition, Characterization, Property, Problem, Example, ExamplesandDefinitions, Hypothesis, Remark, Definition
%% For proofs, please use the proof environment (the amsthm package is loaded by the MDPI class).

%=================================================================
% Full title of the paper (Capitalized)
\Title{Czech and Slovak members of religious institutes their health in
comparison to general population}

% Authors, for the paper (add full first names)
\Author{Dana
Jaksicova$^{1,\ddagger,}$\href{https://orcid.org/0000-0001-6087-9545}{\orcidicon}, Lukas
Novak$^{1,\ddagger,*}$\href{https://orcid.org/0000-0002-7582-2098}{\orcidicon}, Vit
Husek$^{1,\ddagger,}$\href{https://orcid.org/0000-0001-6989-2383}{\orcidicon}, Peter
Tavel$^{1,\ddagger,}$\href{https://orcid.org/0000-0001-7072-001X}{\orcidicon}, Klara
Malinakova$^{1,\ddagger,}$\href{https://orcid.org/0000-0001-6939-1204}{\orcidicon}}

% Authors, for metadata in PDF
\AuthorNames{Dana Jaksicova, Lukas Novak, Vit Husek, Peter Tavel, Klara
Malinakova}

% Affiliations / Addresses (Add [1] after \address if there is only one affiliation.)
\address{%
$^{1}$ \quad Palacky University Olomouc - Social Health Institute
Correspondence Univerzitni 244/22, 771 11, Olomouc, Czech
Republic; \href{mailto:lukas.novak@oushi.upol.cz}{\nolinkurl{lukas.novak@oushi.upol.cz}}\\
}
% Contact information of the corresponding author
\corres{Correspondence: \href{mailto:lukas.novak@oushi.upol.cz}{\nolinkurl{lukas.novak@oushi.upol.cz}};
Tel.: +420-737-823-971.}

% Current address and/or shared authorship

\secondnote{These authors contributed equally to this work.}






% The commands \thirdnote{} till \eighthnote{} are available for further notes

% Simple summary

% Abstract (Do not insert blank lines, i.e. \\)
\abstract{The study examines the general health of consecrated persons
(CP) in the Czech Republic (CZ) and in Slovakia (SK) compared to control
samples of Czech population. Compared to the previous studies from some
western countries, we expected a higher level of adverse health
affecting factors among CP in post-communist CZ and SK. The sample of
293 CP participants (age: M = 47.52, SD = 9.57, females: 78.88\%, 180
Czechs, 213 Slovaks) was compared with two control samples, one was
nationally representative. Comparing CP with general population, we
measured the frequency of recent health complains, the occurrence of
chronic illnesses, the general health and the individual chronotype. CP
compared to the representative sample had a higher chance of suffering
from small pelvis pain and obesity, but a lower chance of diabetes.
Further, CP had higher odds of having worse general health. Comparing
``larks'' with ``night owls'' among CP, ``night owls'' had a
significantly higher chance of suffering from worse general health.
``Night owl CP'' also seem to suffer more from backache and
depression/anxiety and to have more problems with falling asleep.
Compared to the major society, CP in CZ and SK tend to have a similar or
a slightly worse general health. The results differ from the US findings
pointing at the positive health effects of spiritual experience and
structured daily routine of CP. Thus, this study shows the importance of
more detailed research on the way of life of Czech and Slovak CP to
determine the factors with the most negative health effects.}

% Keywords
\keyword{spirituality; consecrated person; health; religious institutes;
Czech; Slovak}

% The fields PACS, MSC, and JEL may be left empty or commented out if not applicable
%\PACS{J0101}
%\MSC{}
%\JEL{}

%%%%%%%%%%%%%%%%%%%%%%%%%%%%%%%%%%%%%%%%%%
% Only for the journal Diversity
%\LSID{\url{http://}}

%%%%%%%%%%%%%%%%%%%%%%%%%%%%%%%%%%%%%%%%%%
% Only for the journal Applied Sciences:
%\featuredapplication{Authors are encouraged to provide a concise description of the specific application or a potential application of the work. This section is not mandatory.}
%%%%%%%%%%%%%%%%%%%%%%%%%%%%%%%%%%%%%%%%%%

%%%%%%%%%%%%%%%%%%%%%%%%%%%%%%%%%%%%%%%%%%
% Only for the journal Data:
%\dataset{DOI number or link to the deposited data set in cases where the data set is published or set to be published separately. If the data set is submitted and will be published as a supplement to this paper in the journal Data, this field will be filled by the editors of the journal. In this case, please make sure to submit the data set as a supplement when entering your manuscript into our manuscript editorial system.}

%\datasetlicense{license under which the data set is made available (CC0, CC-BY, CC-BY-SA, CC-BY-NC, etc.)}

%%%%%%%%%%%%%%%%%%%%%%%%%%%%%%%%%%%%%%%%%%
% Only for the journal Toxins
%\keycontribution{The breakthroughs or highlights of the manuscript. Authors can write one or two sentences to describe the most important part of the paper.}

%\setcounter{secnumdepth}{4}
%%%%%%%%%%%%%%%%%%%%%%%%%%%%%%%%%%%%%%%%%%

% Pandoc citation processing

\usepackage{booktabs}
\usepackage{longtable}
\usepackage{array}
\usepackage{multirow}
\usepackage{wrapfig}
\usepackage{float}
\usepackage{colortbl}
\usepackage{pdflscape}
\usepackage{tabu}
\usepackage{threeparttable}
\usepackage{threeparttablex}
\usepackage[normalem]{ulem}
\usepackage{makecell}
\usepackage{xcolor}

\begin{document}
%%%%%%%%%%%%%%%%%%%%%%%%%%%%%%%%%%%%%%%%%%

\hypertarget{introduction}{%
\section{Introduction}\label{introduction}}

Persons who consecrated themselves through their lifelong commitment to
God and to a certain religious institute represent a minority group
belonging mostly to the Roman Catholic Church
\citep{holtz2006geschichte, frank1988geschichte}. This group differs
from the major society in more aspects and its way of life arouses many
questions \citep{cist2019living, elias2019monasticism}. To the main
specifics of living in a religious institute belong the fundamental role
of spirituality \citep{bussing2017validation} and religiosity
\citep{quartier2017monastic}, the hierarchical structured community and
subordination to the authority, the binding statutes and daily routine
\citep{wrobel2018duties, fernandez2019convents}, gender uniformity and
resignation on sexual relationships and family life
\citep{huel2018missionaries}. Expected is also a high level of social
engagement \citep{bickerton2013spiritual} as well as a certain social
isolation and/or self-excluding from the mainstream
\citep{jewdokimow2019transcending, meawad2020sexuality}. Catholic order
members can be regarded as a homogenous population with a number of
common characteristics on individual and socioeconomic level
\citep{bowen2018community}.\\
Research on health among consecrated people (CP) attracts some
attention, but the studies are sparse and rather ambiguous. The majority
of the existing studies brings positive conclusions. The religious order
members were reported to be at lower risk of mental health disorders
\citep{rogowska2020investigating}, more successful in achieving physical
and mental well-being \citep{corwin2014lord} and able to care
systematically for their health \citep{huck1996health}. The prevalence
of positive emotions was observed among the nuns
\citep{skrzypinska2012intricacies} as well as a higher level of personal
happiness \citep{francis2018happiness} and satisfaction at work
engagement \citep{ariza2018work} together with lower tendence to
professional burnout \citep{chirico2020spirituality}. Several studies
chose the religious communities as a sample of people living a
meaningful and cognitive stimulating life and came to the results of
lower risk of Alzheimer disease or dementia among them
\citep{keohane2019nun, schott2019stability}. It corresponds to the
findings about higher ability of consecrated persons to age meaningfully
\citep{corwin2020care} and about their longevity
\citep{weinstein2019autonomous, danner2001positive}. On the contrary, a
few studies point at some deficits in the health practises of religious
order members \citep{meurer1990exploratory}. These deficits can include
a danger of exhausting the spiritual resources through excessive work
engagement \citep{bickerton2013spiritual} or an impact of community
conflicts on the health of individuals \citep{bowen2018community}.
Furthermore, because the daily routine of the religious communities is
usually strongly oriented to getting up early and going to bed early
without considering individual chronotype, we also suppose persons with
a late circadian timing, the so called ``night owls'', being in some
measure handicapped by this regime. As opposite to the ``larks'' with an
early circadian timing, ``night owls'' prefer to stay up and to work
long in the evening and have problems to wake up and to be active early
in the morning \citep{facer2019resetting}. ``Night owls'' living in a
religious community with a structured daily routine might suffer from
lack of sleep, which might have a negative impact on their health
\citep{duns2019sleep}. However, the existing results cannot be fully
generalized, because there is a need of more data from different
countries and social-political contexts. So far, most of the previous
research studies were carried out in the USA, some of them in Poland,
Germany and Italy. In all these countries, in spite of the advanced
secularisation in some of them, the presence and the social engagement
of religious institutes have an established tradition and a high level
of public credit, which seems to play an important role
\citep{rogowska2020investigating}. On the contrary, data from secular
post-communist countries are missing, as well as data from African,
Asian or South American communities. We presume there is also some other
bias to consider, such as a high level of social desirability and
non-representative samples because all community members rarely complete
the voluntary questionnaires and the healthier and more active persons
are more likely to participate. The current study examines the general
health characteristics of religious brothers and sisters in former
Czechoslovakia compared to control samples of the Czech population. As
the country with the highest percentage (76,4\%) of religiously
unaffiliated people in the world (Pew Research Center 2014), the Czech
Republic (CZ) represents a not very supportive milieu towards the Roman
Catholic Church and religious institutes. Slovakia (SK) belongs
traditionally to the Catholic countries (62\% Roman Catholics and 13,4\%
atheists in the 2011 census) and the consecrated persons are more
numerous and more appreciated than in the Czech Republic. Nevertheless,
the modern history of these two countries is firmly connected and the
convents and monasteries in both of them are still carrying the
consequences of the long persecution, overworking and forced isolation
during the communist regime. Compared to the previous studies, given
these facts, we may expect a higher level of adverse health affecting
factors and lower scores of well-being and life satisfaction among the
members of the Czech and Slovak religious communities, presuming some
slight differences between the Czechs and Slovaks. Therefore, the aim of
this study is to explore the possible associations between membership in
a convent or monastery religious community in two post-communist
countries and the general health characteristics of the individuals.
Further, we will assess if the individual chronotype can play some role
in health of consecrated persons.

\hypertarget{methods}{%
\section{Methods}\label{methods}}

\hypertarget{measures}{%
\subsection{Measures}\label{measures}}

\hypertarget{health-complaints}{%
\subsubsection{Health complaints}\label{health-complaints}}

The frequency of recent health complaints was assessed by the 6-item
measure: headache, stomachache, backache, intestinal problems, trouble
falling asleep, dizziness. The question was: ``In the past month, how
often have you had the following issues?{}`` Each item was answered on a
five-point scale: never (1), about once or twice (2), approximately once
a week (3), more than once a week (4), every day (5). For analytical
purposes, participants responses were dichotomised. Answers ranging from
1:''never" up to 3: ``approximately once a week'' were recoded as ``Not
many times per week'' and answers ranging from 4: ``More than once a
week'' up to 5: ``every day'' were recoded as ``Many times a week''.

\hypertarget{long-lasting-illnesses}{%
\subsubsection{Long lasting illnesses}\label{long-lasting-illnesses}}

For the occurrence of chronic illnesses, an 18-item measure was used
introduced by the question: ``Do you have a long-term illness or
disability? Please tick all that apply to you.`` The illnesses are
listed in table 3.

\hypertarget{general-health}{%
\subsubsection{General health}\label{general-health}}

The GH was assessed by composite variable created by summing up number
of chronic illnesses. This variable was consequently divided into
several categories based on the following approch: having \textless{} 1
disease was classified as ``no diseases'', 1-2 diseases was classified
as ``few diseases'' 3-5 diseases as ``several diseases'' and more than 6
as ``many diseases''.

\hypertarget{chronotype}{%
\subsubsection{Chronotype}\label{chronotype}}

Daily energy was measured by choosing between two possibilities -- an
early bird who wakes up early but is tired in the evening, or a night
owl who has problem with waking up early but enjoys working in the
evening.

\hypertarget{participants}{%
\subsection{Participants}\label{participants}}

\hypertarget{sample-one}{%
\subsubsection{Sample one}\label{sample-one}}

First sample (\emph{n} = 1800, Age: \emph{M} = 46.41, \emph{SD} = 17.4,
Females: 51.28\%) consisted of participants from Czech nationally
representative sample of the study on health, life experience, attitudes
and lifestyle collected in 2016 \citep{malinakova2020religiosity}. In
this dataset we did not find subjects responding incongruently to the
control items i.e.~feeling the God presence despite being non-religious
or atheist. Thus, no participant was excluded from the dataset.

\hypertarget{sample-two}{%
\subsubsection{Sample two}\label{sample-two}}

The second sample was collected in April 2020. It was a survey made in
the Czech population during the first Covid 19-lockdown. From the
original dataset (\emph{n} = 1263), we excluded 120 participants who
responded incongruently to 3 repeatedly asked questions and those, who
were speeders i.e.~time spent on filling the questionnaire was
\textless{} 10 min. The three control questions included age (difference
\textgreater{} 2 years), weight and height (difference \textgreater{} 2
kilogram and centimetres). Hence, the number of participants was 1143.
Based on the results of outliers screening procedure (see statistical
analysis section), we also removed subjects, which responded to large
number of questions in the same way (\emph{n} = 2). Therefore, the final
number of participants was 1141 (Age: \emph{M} = 49.2, \emph{SD} =
16.73, Females: 46.45\%).

\hypertarget{sample-three}{%
\subsubsection{Sample three}\label{sample-three}}

A sample of Catholic order members in the Czech Republic and in the
Slovak Republic was recruited to take part in a survey regarding various
aspects of today´s consecrated life. The respondents were recruited by
contacting the major superiors of all male and female religious
institutes in both countries. After six weeks, the information about the
survey was sent directly into the local communities in order to increase
the number of respondents. The research was done under the auspices of
the Conference of Major Religious Superiors of the Czech and Slovak
Republic. The superiors were asked to spread an online or a
paper-and-pencil questionnaire among the members of their communities
and to support its completing. Data was collected from March to May
2021. This sample initially consisted of 497 participants. In the first
step, we excluded participants (n = 4) who were classified as speeders
i.e.~finished questionnaire typically lasting more than 30 minutes in
\textless{} 10 minutes. After this exclusion, 493 participants remained.
We also removed participants who filled questionnaire multiple times
(\emph{n} = 63) resulting in (n = 430) of subjects . This sample
consisted of 180 Czech participants and 213 Slovak participants. Other
than Czech and Slovak participants were excluded (n = 37) resulting in
393 subjects (Age: M = 47.52, SD = 9.57, Females: 78.88\%). No uniform
pattern of responding was detected in this sample. The mean duration of
being part of a religious community was 24.45.

\hypertarget{statistical-analysis}{%
\subsection{Statistical analysis}\label{statistical-analysis}}

As suggested by Shapiro-Wilk test and by histograms, normality
assumption was broken in all samples. Thus, non - parametric methods
were used. Homogenity of variances was equal in all samples as indicated
by the Breusch-Pagan test. As the null hypotheses of the MCAR test in
all our surveys was not rejected, we deleted missing values listwise.
Outliers were explored by the Median Absolute Deviation (MAD). Outliers
identified by the MED were consequently screened and if there were signs
of uniform pattern of responding i.e.~answering the number of items in
the same manner, than outlier were removed from the dataset.

To explore differences in health status among clerics and non - clerics,
we compared in logistic regression models long lasting illnesses of
clerics to chronic illnesses of participants from representative sample.
In these models, long lasting illness were set as dependent variables.
Grouping variable distinguishing clerics from non - clerics was a
regressor. Covariates consisted of gender, education and age. Ordinal
logistic regression was used to compare clerincs and non - clerics in
the GH. The same regressin type was applied to explore associations
between chronotype and GH in clerics. In the orinal regression models,
the following variables were controled: age, education, gender and
length of a life in clerical order. The Brant test indicated that
proportional odds assumption holded for each of the orinal regression
models. Variance inflation factor (VIF) used to assess multicolinearity
in all regression models. The VIF values \textless{} 10 indicated
acceptable degree association between variables \citep{TAY2017}.
Bonferroni correction was used to correct p-values in all regression
models. When the significnance was lost after correction, we used the
term ``trends'' to describe relationships after correction. The R
\citep[Version 4.0.3;][]{R-base} was utilized for all analysis.

\hypertarget{results}{%
\section{Results}\label{results}}

The table 1 depicts the basic socio-demographic characteristics of the
study samples.

\newpage

\begin{table}

\caption{\label{tab:socio-demographic table}Socio-demographic table}
\centering
\resizebox{\linewidth}{!}{
\begin{tabular}[t]{llll}
\toprule
\multicolumn{1}{c}{ } & \multicolumn{1}{c}{Sample 1} & \multicolumn{1}{c}{Sample 2} & \multicolumn{1}{c}{Sample 3 (CZ,SK)} \\
\cmidrule(l{3pt}r{3pt}){2-2} \cmidrule(l{3pt}r{3pt}){3-3} \cmidrule(l{3pt}r{3pt}){4-4}
Characteristic & N = 1,800 & N = 1,141 & N = 393\\
\midrule
Gender &  &  & \\
\hspace{1em}Female & 923 (51\%) & 530 (50\%) & 310 (79\%)\\
\hspace{1em}Male & 877 (49\%) & 523 (50\%) & 83 (21\%)\\
Family\_status &  &  & \\
\hspace{1em}Not in relationship & 439 (24\%) & 267 (25\%) & \\
\addlinespace
\hspace{1em}Married & 929 (52\%) & 461 (44\%) & \\
\hspace{1em}Divorced & 158 (8.8\%) & 201 (19\%) & \\
\hspace{1em}Widow/Widower & 133 (7.4\%) & 73 (6.9\%) & \\
\hspace{1em}In relationship & 141 (7.8\%) & 51 (4.8\%) & \\
Education &  &  & \\
\addlinespace
\hspace{1em}Basic school & 141 (7.8\%) & 90 (8.7\%) & 1 (0.3\%)\\
\hspace{1em}Vocational school or non - maturity high school & 442 (25\%) & 400 (39\%) & 12 (3.1\%)\\
\hspace{1em}High school & 854 (47\%) & 377 (36\%) & 48 (12\%)\\
\hspace{1em}Higher vocational school or University & 363 (20\%) & 169 (16\%) & 332 (84\%)\\
Economical\_status &  &  & \\
\addlinespace
\hspace{1em}Without work & 261 (14\%) & 149 (14\%) & \\
\hspace{1em}Pensioner & 430 (24\%) & 325 (31\%) & \\
\hspace{1em}Working & 1,109 (62\%) & 559 (54\%) & \\
Faith &  &  & \\
\hspace{1em}Yes, I am a member of church & 170 (9.4\%) &  & \\
\addlinespace
\hspace{1em}Yes, but I am not a member of a church & 361 (20\%) &  & \\
\hspace{1em}No & 1,004 (56\%) &  & \\
\hspace{1em}No, I am convinced atheist & 265 (15\%) &  & \\
\bottomrule
\end{tabular}}
\end{table}

\newpage

\hypertarget{chronic-illness-differences}{%
\subsection{Chronic illness
differences}\label{chronic-illness-differences}}

In the table 2, prevalence of chronic diseases among the study samples
can be found. The table 3 presents differences in chronic diseases
between consecrated persons and the representative sample. It was
revealed that there is a significant positive relationship between being
a consecrated person and lower probability of chronic illnesses such as
diabetes in crude effect. However, there was positive relationship
between being a consecrated person and obesity in crude and adjusted
effect, pain in the small pelvis in both crude and adjusted effect.
After Bonferroni correction, no further significant results were found.

\begin{table}[!h]

\caption{\label{tab:prevalence table}General health and chronic ilnesses among study samples}
\centering
\resizebox{\linewidth}{!}{
\fontsize{8}{10}\selectfont
\begin{tabular}[t]{llll}
\toprule
Characteristic & Sample 1, N = 1,800 & Sample 2, N = 1,141 & Sample 3, N = 393\\
\midrule
ICHS & 68 (3.8\%) & 47 (4.8\%) & 10 (2.9\%)\\
Hypertension & 371 (21\%) & 243 (25\%) & 69 (20\%)\\
Stroke & 20 (1.1\%) & 20 (2.0\%) & 3 (0.9\%)\\
Astma & 166 (9.2\%) & 94 (9.6\%) & 32 (9.4\%)\\
Cancer & 36 (2.0\%) & 28 (2.9\%) & 9 (2.7\%)\\
\addlinespace
Diabetes & 182 (10\%) & 117 (12\%) & 12 (3.5\%)\\
Obesity & 183 (10\%) & 218 (22\%) & 56 (17\%)\\
Arthritis & 121 (6.7\%) & 102 (10\%) & 29 (8.6\%)\\
Back pain & 631 (35\%) & 348 (35\%) & 131 (39\%)\\
Gastric or duodenal ulcers & 56 (3.1\%) & 31 (3.2\%) & 12 (3.5\%)\\
\addlinespace
Chronic lung disease & 24 (1.3\%) & 36 (3.7\%) & 7 (2.1\%)\\
Skin diseases eczema & 156 (8.7\%) & 102 (10\%) & 38 (11\%)\\
Allergy & 364 (20\%) & 178 (18\%) & 83 (24\%)\\
Migraine & 223 (12\%) & 94 (9.6\%) & 42 (12\%)\\
Pain of unclear origin & 99 (5.5\%) & 65 (6.6\%) & 14 (4.1\%)\\
\addlinespace
Pain in the small pelvis & 68 (3.8\%) & 35 (3.6\%) & 37 (11\%)\\
Depression/Anxiety & 125 (6.9\%) & 102 (10\%) & 40 (12\%)\\
Thyroid disease & 152 (8.4\%) & 110 (11\%) & 46 (14\%)\\
General\_health & 1.69 (1.54) & 2.01 (1.93) & 1.98 (1.71)\\
\bottomrule
\multicolumn{4}{l}{\rule{0pt}{1em}\textsuperscript{1} ICHS = Ischemic heart disease, In the General helath variable, values refers to M(SD)}\\
\end{tabular}}
\end{table}

\[\\[1in]\]

\begin{table}[!h]

\caption{\label{tab:Print regres tab 1}Depicts associations (in Odds rations) between living in clerical life and chronic deseases (Sample 1 and 3)}
\centering
\resizebox{\linewidth}{!}{
\fontsize{7}{9}\selectfont
\begin{threeparttable}
\begin{tabular}[t]{llllll}
\toprule
 & Pain in the small pelvis & Obesity & Diabetes & Arthritis & Thyroid disease\\
\midrule
Crude effect & 3.12*** (2.04, 4.72) & 1.75*** (1.25, 2.41) & 0.33*** (0.17, 0.57) & 1.30 (0.84, 1.95) & 1.70** (1.19, 2.40)\\
Adjusted effect & 1.99* (1.16, 3.40) & 1.85** (1.23, 2.78) & 0.46* (0.23, 0.86) & 1.62 (0.94, 2.74) & 1.57* (1.01, 2.45)\\
 & Depression/Anxiety & Migraine & Pain of unclear origin & Cancer & \\
Crude effect & 1.79** (1.22, 2.59) & 1.00 (0.70, 1.41) & 0.74 (0.40, 1.27) & 1.34 (0.60, 2.68) & \\
Adjusted effect & 1.58 (0.98, 2.53) & 0.74 (0.49, 1.12) & 0.77 (0.38, 1.46) & 1.69 (0.64, 4.21) & \\
 & Hypertension & Ischemic heart disease & Stroke & Back pain & \\
Crude effect & 0.98 (0.73, 1.31) & 0.77 (0.37, 1.45) & 0.79 (0.19, 2.33) & 1.17 (0.92, 1.48) & \\
Adjusted effect & 1.26 (0.89, 1.79) & 1.72 (0.72, 3.85) & 1.91 (0.36, 8.45) & 1.27 (0.95, 1.70) & \\
 & Gastric or duodenal ulcers & Chronic lung disease & Skin diseases eczema & Allergy & \\
Crude effect & 1.14 (0.58, 2.08) & 1.56 (0.62, 3.47) & 1.33 (0.90, 1.92) & 1.28 (0.97, 1.67) & \\
Adjusted effect & 1.28 (0.57, 2.74) & 1.51 (0.50, 4.29) & 1.31 (0.82, 2.07) & 1.15 (0.83, 1.60) & \\
\bottomrule
\end{tabular}
\begin{tablenotes}[para]
\item \textit{Note:} 
\item p < 0.05 *, p < 0.01 ** p < 0.001 ***, Adjusted effect was calculated using the following variables as a covariates: Age, Gender and Education. Values in brackets indicates 95\% confidence interval. After Bonferroni correction all results in the first row (execpt Thyroid disease) remained significant. Other relationships were non - significant.
\end{tablenotes}
\end{threeparttable}}
\end{table}

\[\\[1in]\]

\[\\[1in]\]

\hypertarget{health-complains}{%
\subsection{Health complains}\label{health-complains}}

Table 4 refers to associations between health complains in CP as
compared to sample 2. Logistic regression indicated a significant
relationship between CP and sleep problems: in crude effect, CP had a
lower odds of having trouble falling asleep. No further significant
associations were found.

\begin{table}[!h]

\caption{\label{tab:Print regres tab 2}Depicts associations (in Odds rations) between living in clerical life and health complains in the last month (Sample 2 and 3)}
\centering
\resizebox{\linewidth}{!}{
\fontsize{7}{9}\selectfont
\begin{threeparttable}
\begin{tabular}[t]{llll}
\toprule
 & Trouble falling asleep & Headache & Stomachache\\
\midrule
Crude effect & 0.65* (0.45, 0.92) & 0.87 (0.54, 1.36) & 0.68 (0.33, 1.28)\\
Adjusted effect & 0.84 (0.52, 1.34) & 0.99 (0.53, 1.83) & 0.61 (0.25, 1.43)\\
 & Backache & Intestinal problems & Dizziness\\
Crude effect & 0.99 (0.74, 1.31) & 1.31 (0.78, 2.16) & 0.85 (0.41, 1.64)\\
Adjusted effect & 1.00 (0.67, 1.48) & 1.37 (0.67, 2.81) & 1.41 (0.53, 3.60)\\
\bottomrule
\end{tabular}
\begin{tablenotes}[para]
\item \textit{Note:} 
\item p < 0.05 *, p < 0.01 ** p < 0.001 ***, Adjusted effect was calculated using the following variables as a covariates: Age, Gender and Education. Values in brackets indicates 95\% confidence interval.
\end{tablenotes}
\end{threeparttable}}
\end{table}

\hypertarget{general-health-1}{%
\subsection{General health}\label{general-health-1}}

Ordinal logistic regression revealed that clerics being night owls had
significantly higher chance of lower GH as compared to early birds - in
crude effect (OR 1.53; 95\% CI (1.02, 2.30); p=0.039). In the adjusted
effect however, this result was non - significant (OR 1.45; 95\% CI
(0.96, 2.21); p=0.078).

In the next step, we compared CP with the representative sample in the
general health. It was found that CP had significantly higher odds of
having lower general health in crude effect (OR 1.36; 95\% CI (1.09,
1.69); p=0.007). Moreover, in the adjusted effect, the odds of having
lower general health slightly increased (OR 1.39; 95\% CI (1.07, 1.81);
p=0.013).

\hypertarget{chronotype-and-health-complains}{%
\subsection{Chronotype and health
complains}\label{chronotype-and-health-complains}}

The table 5 shows associations between early birds and night owls in
health complains in CP sample. The following trends were found after
Bonferroni correction: night owls had higher probability of suffering
from problems of falling asleep and backache as compared to early birds
(in both crude and adjusted effect).

\begin{table}[!h]

\caption{\label{tab:Print regres tab 5}Depicts associations (in Odds rations) between eary bird and night owns and health complains (Sample 3)}
\centering
\resizebox{\linewidth}{!}{
\fontsize{7}{9}\selectfont
\begin{threeparttable}
\begin{tabular}[t]{llll}
\toprule
 & Headache & Stomachache & Backache\\
\midrule
Crude effect & 0.76 (0.32, 1.70) & 0.70 (0.18, 2.37) & 1.73* (1.05, 2.86)\\
Adjusted effect & 0.72 (0.31, 1.36) & 0.70 (0.18, 2.42) & 1.74* (1.04, 2.92)\\
 & Intestinal problems & Trouble falling asleep & Dizziness\\
Crude effect & 1.67 (0.71, 4.02) & 2.55** (1.34, 4.99) & 1.51 (0.45, 5.33)\\
Adjusted effect & 1.60 (0.68, 3.91) & 2.59** (1.35, 5.11) & 1.44 (0.42, 5.12)\\
\bottomrule
\end{tabular}
\begin{tablenotes}[para]
\item \textit{Note:} 
\item p < 0.05 *, p < 0.01 ** p < 0.001 ***, Adjusted effect was calculated using the following variables as a covariates: Age, Gender, Education and yeas being in clerical order. Values in brackets indicates 95\% confidence interval. After Bonferroni correction all results were non - significant.
\end{tablenotes}
\end{threeparttable}}
\end{table}

\hypertarget{chronotype-and-chronic-ilnessess}{%
\subsection{Chronotype and chronic
ilnessess}\label{chronotype-and-chronic-ilnessess}}

Table 6 represents results of logistic regression comparing CP being
night owls to CP being early birds. Although after Bonferroni correction
all results were non-significant, several trends might be observed:
night owls as compared to early birds had higher odds of developing
chronic arthritis (adjusted effect) and anxiety/depression (crude and
adjusted effect).

\begin{table}[!h]

\caption{\label{tab:Print regres tab 6}Depicts associations (in Odds rations) between eary bird and night owns clerics and chronical mental and physical deseases (Sample 3)}
\centering
\resizebox{\linewidth}{!}{
\fontsize{7}{9}\selectfont
\begin{threeparttable}
\begin{tabular}[t]{llllll}
\toprule
 & Gastric or duodenal ulcers & Chronic lung disease & Skin diseases eczema & Allergy & Migraine\\
\midrule
Crude effect & 2.23 (0.66, 8.64) & 1.67 (0.36, 8.60) & 1.27 (0.65, 2.52) & 0.93 (0.56, 1.54) & 1.15 (0.60, 2.20)\\
Adjusted effect & 2.12 (0.63, 8.27) & 1.66 (0.36, 8.55) & 1.28 (0.64, 2.55) & 0.94 (0.57, 1.57) & 1.04 (0.53, 2.01)\\
 & Depression/Anxiety & Ischemic heart disease & Obesity & Back pain & \\
Crude effect & 2.02* (1.04, 4.03) & 0.82 (0.21, 2.93) & 1.41 (0.79, 2.52) & 1.02 (0.66, 1.59) & \\
Adjusted effect & 2.00* (1.03, 4.01) & 0.54 (0.11, 2.30) & 1.47 (0.81, 2.67) & 1.01 (0.64, 1.60) & \\
 & Hypertension & Diabetes & Arthritis & Astma & \\
Crude effect & 1.26 (0.74, 2.15) & 0.88 (0.26, 2.82) & 2.55* (1.17, 5.89) & 1.93 (0.93, 4.14) & \\
Adjusted effect & 1.25 (0.71, 2.18) & 0.86 (0.25, 2.78) & 3.49** (1.43, 9.31) & 1.89 (0.90, 4.08) & \\
 & Pain of unclear origin & Pain in the small pelvis & Cancer & Thyroid disease & \\
Crude effect & 0.32 (0.07, 1.06) & 0.64 (0.31, 1.29) & 0.35 (0.05, 1.46) & 1.16 (0.62, 2.17) & \\
Adjusted effect & 0.36 (0.08, 1.21) & 0.63 (0.29, 1.30) & 0.29 (0.04, 1.34) & 1.03 (0.54, 1.95) & \\
\bottomrule
\end{tabular}
\begin{tablenotes}[para]
\item \textit{Note:} 
\item p < 0.05 *, p < 0.01 ** p < 0.001 ***, Adjusted effect was calculated using the following variables as a covariates: Age, Gender, Education and number of years being part of the clerical order. Values in brackets indicates 95\% confidence interval. After Bonferroni correction all results were non - significant. Variable stroke was excldued from analysis, because regression model containing this variable not converge.
\end{tablenotes}
\end{threeparttable}}
\end{table}

\hypertarget{discussion}{%
\section{Discussion}\label{discussion}}

The aim of the study was to assess the relation between living as a
consecrated person in religious institutes in the Czech Republic and in
the Slovak Republic and general health. Compared to a Czech national
representative sample, the results showed a lower probability of
suffering from diabetes in CP. However, CP were found to be in a higher
risk of obesity and pain in the small pelvis (women CPs). Further, we
discovered no significant results in health complaints of CP in
comparison with the control sample. Only the item ``trouble falling
asleep'' in the religious sample was close to the significance
threshold. Moreover, CP had significantly more chronic diseases. More
specifically, when focusing only on CP, we found a higher risk of
suffering from chronic illnesses for ``night owls'' compared to
``larks'', though only in crude effect. ``Night owl CP'' seem to suffer
more from arthritis, backache and depression/anxiety and have more
problems with falling asleep. The association of being a consecrated
person with higher risk of obesity (table 2) can be explained by several
reasons. According to some findings, higher tendency to obesity is
observed among believers in general \citep{koenig2012religion}. Further,
although there are no specialised studies in this social group, some
general findings can be applied considering the reality of religious
communities. Regarding the high-performance orientation, enrooted for
many decades particularly in apostolic religious congregations
\citep{jakvsivcova2021sluzebnici}, we can suppose the lack of sport and
physical exercise as well as the lack of sleep, which is often connected
with unhealthy eating habits \citep{narcisse2018mediating}. Furthermore,
some religious communities still tend to follow the old uniforming rules
which do not support individual diets \citep{jakvsivcova2012dcery}. In
this case, all members are supposed to eat the same food, which may not
be appropriate for everybody. The performance pressure is also related
with higher level of stress as another factor leading to unregulated
eating \citep{ling2021relationships, leow2021understanding} and
potentially elevated levels of cortisol, which is further influencing
one´s metabolism \citep{kistenmacher2018psychosocial}. On the contrary,
though a higher risk of diabetes could be expected among CP due to the
already described higher risk of obesity, which belongs to the main risk
factors of diabetes \citep{brovz2020prevalence}, our results did not
support this presumption. These findings may be explained by the fact
that another factor significantly contributing to the development of
diabetes is smoking \citep{brovz2020prevalence}, which is, however,
quite rare among CP. Women CPs are non-smokers \citep{butler1996trends}
and there is only a small group of smokers among priest and men CPs
\citep{stang2012cancer}. However, more data would be needed to support
this potential explanation. Moreover, we found two or three times higher
risk of the small pelvis pain among religious sisters. There might be
several reasons for this. Life in celibacy includes a higher risk of
psychosomatic complaints caused by libido-suppression and of
psychosexual problems \citep{baumann2019spiritual}. Further, we can
argue that issues regarding sexuality are in some extent still taboo in
religious institutes, because this approach is deeply enrooted in the
mentality of elderly sisters, which also tends to self-denial and
ignoring of health problems \citep{jakvsivcova2021sluzebnici}.
Therefore, it may be challenging for some consecrated women to visit
gynaecologist regularly and to solve the starting complaints soon
enough. According to the latest findings of Nygaard et
al.~\citep{nygaard2020baseline}, in case of chronic pelvic pains, early
intervention is important to reduce the complaints successfully. It
remains a task for the future research to study the extent of traumatic
experiences with sexual abuse among CP compared with a sample of
non-CPs, because it might be another possible reason for the chronic
small pelvis pain in CP \citep{heim1998abuse}. Our finding of a lower
general tendency of CP to have sleep disturbance remained close to
significance threshold and more data would be needed to confirm it.
However, religious community members with ``night owl'' chronotype
reported sleep disturbances significantly more often than ``lark CP''.
Looking only at the religious sample, ``night owls'' declared evidently
higher probability to have troubles falling asleep than ``larks''. A
common daily routine in a religious community is strictly oriented to
getting up early and going to bed early, which is favourable for
``larks'' and inconvenient for ``owls'', who have to adapt to this
routine. In some extent, it may be useful for them, because shifting
sleep/wake timings in ``night owls'' to earlier hours showed positive
effects on their performance and mental health, as reported by
Facer-Childs \citep{facer2019resetting}. However, the biorhythm still
might cause difficulties to fall asleep early. As a consequence, some
``owl-brothers/owl sisters'' seem to give up going to bed early and to
continue working late in the evening according to their chronotype, but
they have to respect the regime given and get up early to attend the
community morning prayer. It may result in the chronic lack of sleep in
this group of consecrated persons, possibly leading to other health
issues. Several studies exploring the individual chronotype concluded,
that the definite evening types -- ``night owls'' show higher rates of
metabolic dysfunction and cardiovascular disease and are in higher
mortality risk \citep{knutson2018associations}. Our sample is too small
to discuss this topic, however, it showed some results about evening
types having worse health characteristics. The revealed connection
between being a ``night owl CP'' and a higher risk of depression is
consistent with the general findings about this chronotype
\citep{facer2019resetting}. Furthermore, these CP may be stressed by the
permanent lack of sleep, which can contribute to a development of
depression. The higher backache tendency in ``night owl CP'' may be a
psychosomatic consequence of a higher stress level
\citep{hajnovic2018causal}. A higher risk of arthritis in ``night owl
CP'' does not correspond with the previous findings of Habers et
al.~about rheumatoid arthritis patients showing an earlier circadian
rhythm \citep{habers2021earlier}. However, our sample is not large
enough and we would need more data to check up this potential
assotiation. When comparing Czech and Slovak CP with the Czech
representative sample regarding the general health, we found
significantly higher risk of having worse general health for CP.
Moreover, these findings are supported by a trend, however
insignificant, that could be observable in most of the chronic diseases
in our study. These findings correspond to our hypothesis that comparing
to the results from USA
\citep{schott2019stability, weinstein2019autonomous}, there are more
negative health-effecting factors in CP in CZ and SK. It seems to be
connected with different mentality and lifestyle of CP in the USA and in
CZ/SK, influenced by the major society and the historical, cultural and
ecclesiastical background, especially with delayed and just partial
reception of postconciliar changes after Vaticanum II, which is typical
for the post-communist countries
\citep{fiala2000koncil, balik2012letnice}. The Czech/Slovak CP still
seem to tend to the traditional performance-oriented lifestyle
characterised by strict self-denial \citep{petravcek2013sekularizace}.
Thus, possibly reasons for the worse general health in Czech/Slovak CP
could be overworking, stress and lack of active and passive rest caused
by the stereotyped regime of the communities, unadapted to their actual
situation and not giving enough space for leisure and relaxation of
individuals. Perhaps also higher incidence of psychosomatic problems
could be considered. All these are hypotheses for future research.
Strengths and limitations This study has several important strengths,
the most important is that it is the first study about consecrated
persons living in the specific milieu of two post-communist countries.
The study is based on the first blanket research among CP in these
countries, which is covering many dimensions of consecrated life and
allows to investigate various factors of living in a religious
community. Moreover, as a comparion, this research uses a large
nationally representative Czech sample and another large professionally
gathered online sample close to national charcteristics. Nevertheless,
not having nationally representative data from a Slovak sample
represents a certain limitation of this study. We accept that the
numbers could slightly change if these would be included, as there are
slight differences between Czechs and Slovaks in life expectancy and
prevalence of some chronic illnesses \citep{vagavsova2017disparities}.
Nevertheless, we do not suppose these potential differences changing the
finding that consecrated persons in CZ and SK in our sample seem to have
similar or worse general health compared to the major population.
Another limitation of our study is related to the fact that the sample
of CP is relatively small because of the general mistrust of CP towards
surveys asking personal questions. We can also expect a certain
selection bias in the sample, supposing the more active, more interested
and healthier CP more likely to fill in the questionnaire. However, in
case of this study, this selection bias seems rather to affirm the
conclusion about the lower general health of CP. Implications Our
findings suggest that there is a need for researchers and also for
spiritual directors to focus more deeply on internal structures, daily
routine, relationships, working habits and mental hygiene in religious
communities. These issues should also be discussed directly in the
religious communities and in the courses for religious formators.
Furthermore, doctors, psychologists, counsellors and other helping
professions should be informed more about the lifestyle of consecrated
persons. Future research should be oriented to revealing positive and
negative health effecting factors among CP, comparing CP both from local
communities of one country and from different countries and cultural
contexts, so that generally spread and culturally conditioned factors
can be distinguished.

\hypertarget{conclusion}{%
\section{Conclusion}\label{conclusion}}

According to our findings, compared to the major society, persons living
the consecrated life in the post-communist Czech Republic and Slovak
Republic tend to have a similar or a slightly worse general health. The
results differ from the US findings pointing at the positive health
effects of spiritual experience and structured daily routine. Thus, this
study shows the importance of more detailed research in the way of life
of (not only) Czech and Slovak religious communities to determine
factors that could be contributing to their negative health outcomes.
Ethics statement The study design was approved by the Ethics Committee
of the Olomouc University Social Health Institute (No.~2021/3).

\hypertarget{patents}{%
\section{Patents}\label{patents}}

% %%%%%%%%%%%%%%%%%%%%%%%%%%%%%%%%%%%%%%%%%%
% %% optional
% \supplementary{The following are available online at www.mdpi.com/link, Figure S1: title, Table S1: title, Video S1: title.}
%
% % Only for the journal Methods and Protocols:
% % If you wish to submit a video article, please do so with any other supplementary material.
% % \supplementary{The following are available at www.mdpi.com/link: Figure S1: title, Table S1: title, Video S1: title. A supporting video article is available at doi: link.}

\vspace{6pt}

%%%%%%%%%%%%%%%%%%%%%%%%%%%%%%%%%%%%%%%%%%
\acknowledgments{All sources of funding of the study should be
disclosed. Please clearly indicate grants that you have received in
support of your research work. Clearly state if you received funds for
covering the costs to publish in open access.}

%%%%%%%%%%%%%%%%%%%%%%%%%%%%%%%%%%%%%%%%%%
\authorcontributions{For research articles with several authors, a short
paragraph specifying their individual contributions must be provided.
The following statements should be used ``X.X. and Y.Y. conceive and
designed the experiments; X.X. performed the experiments; X.X. and Y.Y.
analyzed the data; W.W. contributed reagents/materials/analysis tools;
Y.Y. wrote the paper.'\,' Authorship must be limited to those who have
contributed substantially to the work reported.}

%%%%%%%%%%%%%%%%%%%%%%%%%%%%%%%%%%%%%%%%%%
\conflictsofinterest{Declare conflicts of interest or state `The authors
declare no conflict of interest.' Authors must identify and declare any
personal circumstances or interest that may be perceived as
inappropriately influencing the representation or interpretation of
reported research results. Any role of the funding sponsors in the
design of the study; in the collection, analyses or interpretation of
data in the writing of the manuscript, or in the decision to publish the
results must be declared in this section. If there is no role, please
state `The founding sponsors had no role in the design of the study; in
the collection, analyses, or interpretation of data; in the writing of
the manuscript, an in the decision to publish the results'.}

%%%%%%%%%%%%%%%%%%%%%%%%%%%%%%%%%%%%%%%%%%
%% optional
\abbreviations{The following abbreviations are used in this manuscript:\\

\noindent
\begin{tabular}{@{}ll}
MDPI & Multidisciplinary Digital Publishing Institute \\
DOAJ & Directory of open access journals \\
TLA & Three letter acronym \\
LD & linear dichroism \\
MSE & Mean Square Error \\
\end{tabular}}

\input{"appendix.tex"}

%%%%%%%%%%%%%%%%%%%%%%%%%%%%%%%%%%%%%%%%%%
% Citations and References in Supplementary files are permitted provided that they also appear in the reference list here.

%=====================================
% References, variant A: internal bibliography
%=====================================
%\reftitle{References}
%\begin{thebibliography}{999}
% Reference 1
%\bibitem[Author1(year)]{ref-journal}
%Author1, T. The title of the cited article. {\em Journal Abbreviation} {\bf 2008}, {\em 10}, 142--149.
% Reference 2
%\bibitem[Author2(year)]{ref-book}
%Author2, L. The title of the cited contribution. In {\em The Book Title}; Editor1, F., Editor2, A., Eds.; Publishing House: City, Country, 2007; pp. 32--58.
%\end{thebibliography}

% The following MDPI journals use author-date citation: Arts, Econometrics, Economies, Genealogy, Humanities, IJFS, JRFM, Laws, Religions, Risks, Social Sciences. For those journals, please follow the formatting guidelines on http://www.mdpi.com/authors/references
% To cite two works by the same author: \citeauthor{ref-journal-1a} (\citeyear{ref-journal-1a}, \citeyear{ref-journal-1b}). This produces: Whittaker (1967, 1975)
% To cite two works by the same author with specific pages: \citeauthor{ref-journal-3a} (\citeyear{ref-journal-3a}, p. 328; \citeyear{ref-journal-3b}, p.475). This produces: Wong (1999, p. 328; 2000, p. 475)

%=====================================
% References, variant B: external bibliography
%=====================================
\reftitle{References}
\externalbibliography{yes}
\bibliography{mybibfile.bib}

%%%%%%%%%%%%%%%%%%%%%%%%%%%%%%%%%%%%%%%%%%
%% optional
\sampleavailability{Study data and R code are availible on Open Science
Framework website {[}accessed date: 10 August 2021{]}
(\url{https://osf.io/tv6m5/}).}

%% for journal Sci
%\reviewreports{\\
%Reviewer 1 comments and authors’ response\\
%Reviewer 2 comments and authors’ response\\
%Reviewer 3 comments and authors’ response
%}

%%%%%%%%%%%%%%%%%%%%%%%%%%%%%%%%%%%%%%%%%%
\end{document}

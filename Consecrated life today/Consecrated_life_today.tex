%  LaTeX support: latex@mdpi.com
%  In case you need support, please attach all files that are necessary for compiling as well as the log file, and specify the details of your LaTeX setup (which operating system and LaTeX version / tools you are using).

%=================================================================
\documentclass[ijerph,article,accept,moreauthors,pdftex]{mdpi}

% If you would like to post an early version of this manuscript as a preprint, you may use preprint as the journal and change 'submit' to 'accept'. The document class line would be, e.g., \documentclass[preprints,article,accept,moreauthors,pdftex]{mdpi}. This is especially recommended for submission to arXiv, where line numbers should be removed before posting. For preprints.org, the editorial staff will make this change immediately prior to posting.

%% Some pieces required from the pandoc template
\providecommand{\tightlist}{%
  \setlength{\itemsep}{0pt}\setlength{\parskip}{4pt}}
\setlist[itemize]{leftmargin=*,labelsep=5.8mm}
\setlist[enumerate]{leftmargin=*,labelsep=4.9mm}

\usepackage{longtable}

% see https://stackoverflow.com/a/47122900

%--------------------
% Class Options:
%--------------------
%----------
% journal
%----------
% Choose between the following MDPI journals:
% acoustics, actuators, addictions, admsci, aerospace, agriculture, agriengineering, agronomy, algorithms, animals, antibiotics, antibodies, antioxidants, applsci, arts, asc, asi, atmosphere, atoms, axioms, batteries, bdcc, behavsci , beverages, bioengineering, biology, biomedicines, biomimetics, biomolecules, biosensors, brainsci , buildings, cancers, carbon , catalysts, cells, ceramics, challenges, chemengineering, chemistry, chemosensors, children, cleantechnol, climate, clockssleep, cmd, coatings, colloids, computation, computers, condensedmatter, cosmetics, cryptography, crystals, dairy, data, dentistry, designs , diagnostics, diseases, diversity, drones, econometrics, economies, education, electrochem, electronics, energies, entropy, environments, epigenomes, est, fermentation, fibers, fire, fishes, fluids, foods, forecasting, forests, fractalfract, futureinternet, futurephys, galaxies, games, gastrointestdisord, gels, genealogy, genes, geohazards, geosciences, geriatrics, hazardousmatters, healthcare, heritage, highthroughput, horticulturae, humanities, hydrology, ijerph, ijfs, ijgi, ijms, ijns, ijtpp, informatics, information, infrastructures, inorganics, insects, instruments, inventions, iot, j, jcdd, jcm, jcp, jcs, jdb, jfb, jfmk, jimaging, jintelligence, jlpea, jmmp, jmse, jnt, jof, joitmc, jpm, jrfm, jsan, land, languages, laws, life, literature, logistics, lubricants, machines, magnetochemistry, make, marinedrugs, materials, mathematics, mca, medicina, medicines, medsci, membranes, metabolites, metals, microarrays, micromachines, microorganisms, minerals, modelling, molbank, molecules, mps, mti, nanomaterials, ncrna, neuroglia, nitrogen, notspecified, nutrients, ohbm, particles, pathogens, pharmaceuticals, pharmaceutics, pharmacy, philosophies, photonics, physics, plants, plasma, polymers, polysaccharides, preprints , proceedings, processes, proteomes, psych, publications, quantumrep, quaternary, qubs, reactions, recycling, religions, remotesensing, reports, resources, risks, robotics, safety, sci, scipharm, sensors, separations, sexes, signals, sinusitis, smartcities, sna, societies, socsci, soilsystems, sports, standards, stats, surfaces, surgeries, sustainability, symmetry, systems, technologies, test, toxics, toxins, tropicalmed, universe, urbansci, vaccines, vehicles, vetsci, vibration, viruses, vision, water, wem, wevj

%---------
% article
%---------
% The default type of manuscript is "article", but can be replaced by:
% abstract, addendum, article, benchmark, book, bookreview, briefreport, casereport, changes, comment, commentary, communication, conceptpaper, conferenceproceedings, correction, conferencereport, expressionofconcern, extendedabstract, meetingreport, creative, datadescriptor, discussion, editorial, essay, erratum, hypothesis, interestingimages, letter, meetingreport, newbookreceived, obituary, opinion, projectreport, reply, retraction, review, perspective, protocol, shortnote, supfile, technicalnote, viewpoint
% supfile = supplementary materials

%----------
% submit
%----------
% The class option "submit" will be changed to "accept" by the Editorial Office when the paper is accepted. This will only make changes to the frontpage (e.g., the logo of the journal will get visible), the headings, and the copyright information. Also, line numbering will be removed. Journal info and pagination for accepted papers will also be assigned by the Editorial Office.

%------------------
% moreauthors
%------------------
% If there is only one author the class option oneauthor should be used. Otherwise use the class option moreauthors.

%---------
% pdftex
%---------
% The option pdftex is for use with pdfLaTeX. If eps figures are used, remove the option pdftex and use LaTeX and dvi2pdf.

%=================================================================
\firstpage{1}
\makeatletter
\setcounter{page}{\@firstpage}
\makeatother
\pubvolume{xx}
\issuenum{1}
\articlenumber{5}
\pubyear{2019}
\copyrightyear{2019}
%\externaleditor{Academic Editor: name}
\history{Received: date; Accepted: date; Published: date}
\updates{yes} % If there is an update available, un-comment this line

%% MDPI internal command: uncomment if new journal that already uses continuous page numbers
%\continuouspages{yes}

%------------------------------------------------------------------
% The following line should be uncommented if the LaTeX file is uploaded to arXiv.org
%\pdfoutput=1

%=================================================================
% Add packages and commands here. The following packages are loaded in our class file: fontenc, calc, indentfirst, fancyhdr, graphicx, lastpage, ifthen, lineno, float, amsmath, setspace, enumitem, mathpazo, booktabs, titlesec, etoolbox, amsthm, hyphenat, natbib, hyperref, footmisc, geometry, caption, url, mdframed, tabto, soul, multirow, microtype, tikz

%=================================================================
%% Please use the following mathematics environments: Theorem, Lemma, Corollary, Proposition, Characterization, Property, Problem, Example, ExamplesandDefinitions, Hypothesis, Remark, Definition
%% For proofs, please use the proof environment (the amsthm package is loaded by the MDPI class).

%=================================================================
% Full title of the paper (Capitalized)
\Title{Consecrated life today. Comparision of general health
characteristics between non clerical and clerical samples}

% Authors, for the paper (add full first names)
\Author{Dana
Jaksicova$^{1,\ddagger,}$\href{https://orcid.org/0000-0001-6087-9545}{\orcidicon}, Klara
Malinakova$^{1,\ddagger,}$\href{https://orcid.org/0000-0001-6939-1204}{\orcidicon}, Lukas
Novak$^{1,\ddagger,*}$\href{https://orcid.org/0000-0002-7582-2098}{\orcidicon}, Peter
Tavel$^{1,\ddagger,}$\href{https://orcid.org/0000-0001-7072-001X}{\orcidicon}}

% Authors, for metadata in PDF
\AuthorNames{Dana Jaksicova, Klara Malinakova, Lukas Novak, Peter Tavel}

% Affiliations / Addresses (Add [1] after \address if there is only one affiliation.)
\address{%
$^{1}$ \quad Palacky University Olomouc - Social Health Institute
Correspondence Univerzitni 244/22, 771 11, Olomouc, Czech
Republic; \href{mailto:dana.jaksicova@oushi.upol.cz}{\nolinkurl{dana.jaksicova@oushi.upol.cz}}\\
}
% Contact information of the corresponding author
\corres{Correspondence: \href{mailto:dana.jaksicova@oushi.upol.cz}{\nolinkurl{dana.jaksicova@oushi.upol.cz}};
Tel.: +XX-000-00-0000.}

% Current address and/or shared authorship
\firstnote{Current address: Updated affiliation}
\secondnote{These authors contributed equally to this work.}






% The commands \thirdnote{} till \eighthnote{} are available for further notes

% Simple summary
\simplesumm{A Simple summary goes here.}

% Abstract (Do not insert blank lines, i.e. \\)
\abstract{dfdfdf}

% Keywords
\keyword{keyword 1; keyword 2; keyword 3 (list three to ten pertinent
keywords specific to the article, yet reasonably common within the
subject discipline.).}

% The fields PACS, MSC, and JEL may be left empty or commented out if not applicable
%\PACS{J0101}
%\MSC{}
%\JEL{}

%%%%%%%%%%%%%%%%%%%%%%%%%%%%%%%%%%%%%%%%%%
% Only for the journal Diversity
%\LSID{\url{http://}}

%%%%%%%%%%%%%%%%%%%%%%%%%%%%%%%%%%%%%%%%%%
% Only for the journal Applied Sciences:
%\featuredapplication{Authors are encouraged to provide a concise description of the specific application or a potential application of the work. This section is not mandatory.}
%%%%%%%%%%%%%%%%%%%%%%%%%%%%%%%%%%%%%%%%%%

%%%%%%%%%%%%%%%%%%%%%%%%%%%%%%%%%%%%%%%%%%
% Only for the journal Data:
%\dataset{DOI number or link to the deposited data set in cases where the data set is published or set to be published separately. If the data set is submitted and will be published as a supplement to this paper in the journal Data, this field will be filled by the editors of the journal. In this case, please make sure to submit the data set as a supplement when entering your manuscript into our manuscript editorial system.}

%\datasetlicense{license under which the data set is made available (CC0, CC-BY, CC-BY-SA, CC-BY-NC, etc.)}

%%%%%%%%%%%%%%%%%%%%%%%%%%%%%%%%%%%%%%%%%%
% Only for the journal Toxins
%\keycontribution{The breakthroughs or highlights of the manuscript. Authors can write one or two sentences to describe the most important part of the paper.}

%\setcounter{secnumdepth}{4}
%%%%%%%%%%%%%%%%%%%%%%%%%%%%%%%%%%%%%%%%%%

% Pandoc citation processing

\usepackage{booktabs}
\usepackage{longtable}
\usepackage{array}
\usepackage{multirow}
\usepackage{wrapfig}
\usepackage{float}
\usepackage{colortbl}
\usepackage{pdflscape}
\usepackage{tabu}
\usepackage{threeparttable}
\usepackage{threeparttablex}
\usepackage[normalem]{ulem}
\usepackage{makecell}
\usepackage{xcolor}

\begin{document}
%%%%%%%%%%%%%%%%%%%%%%%%%%%%%%%%%%%%%%%%%%

\hypertarget{introduction}{%
\section{Introduction}\label{introduction}}

Persons who consecrated themselves through their lifelong commitment to
God and to a certain religious institute represent a minority group
belonging mostly to the Roman Catholic Church
\citep{holtz2006geschichte, frank1988geschichte}. This group differs
from the major society in more aspects and its way of life arouses many
questions {[}\citet{cist2019living};elias2019monasticism{]}. To the main
specifics of living in a religious institute belong the fundamental role
of spirituality \citep{bussing2017validation} and religiosity
\citep{quartier2017monastic}, the hierarchical structured community and
subordination to the authority, the binding statutes and daily routine
{[}\citet{wrobel2018duties};fernandez2019convents{]}, gender uniformity
and resignation on sexual relationships and family life
\citep{huel2018missionaries}. Expected is also a high level of social
engagement \citep{bickerton2013spiritual} as well as a certain social
isolation and/or self-excluding from the mainstream
\citep{jewdokimow2019transcending, meawad2020sexuality}. Catholic order
members can be regarded as a homogenous population with a number of
common characteristics on individual and socioeconomic level
\citep{bowen2018community}.\\
Research on health among consecrated people attracts some attention but
the studies are sparse and rather ambiguous. The majority of the
existing studies brings positive conclusions. The religious order
members might be at lower risk of mental health disorders
\citep{rogowska2020investigating}, more successful in achieving physical
and mental well-being \citep{corwin2014lord} and able to care
systematically for their health \citep{huck1996health}. The prevalence
of positive emotions was observed among the nuns
\citep{skrzypinska2012intricacies} as well as a higher level of personal
happiness \citep{francis2018happiness} and satisfaction at work
engagement \citep{ariza2018work} together with lower tendence to
professional burnout \citep{chirico2020spirituality}. Several studies
chose the religious communities as a sample of people living a
meaningful and cognitive stimulating life and came to the results of
lower risk of Alzheimer disease or dementia among them
\citep{keohane2019nun, schott2019stability}. It corresponds to the
findings about higher ability of consecrated persons to age meaningful
\citep{corwin2020care} and about their longevity
\citep{weinstein2019autonomous, danner2001positive}. On the contrary,
there are very few studies pointing at some deficits in health practises
of religious order members \citep{meurer1990exploratory}, at the danger
of exhausting the spiritual resources through excessive work engagement
\citep{bickerton2013spiritual} or at the impact the community conflicts
have on the health of individuals \citep{bowen2018community}. However,
the existing results cannot be generalized. There is need of more data
from different countries and social-political contexts. Most of the
previous research studies were carried out in the USA, some of them in
Poland, Germany and Italy. In all these countries, in spite of the
advanced secularisation in some of them, the presence and the social
engagement of religious institutes have an established tradition and a
high level of public credit, which seems to play some role
\citep{rogowska2020investigating}. On the contrary, data from secular
post-communist countries are missing as well as data from African, Asian
or South American communities. We presume, there are also some other
bias to consider, such as a high level of social desirability and
non-representative samples because the voluntary questionnaires are
rarely completed by all members of the community. The healthier and more
active persons are more likely to participate. The current study
examines the general health characteristics of religious brothers and
sisters in the Czech Republic in comparison to control samples of the
Czech population and a sample of consecrated persons from the Slovak
Republic. The Czech Republic as the country with the highest percentage
(76,4\%) of religiously unaffiliated people in the world (Pew Research
Center 2014) represents a not very supportive milieu towards the Roman
Catholic Church and religious institutes. Slovakia belongs traditionally
to the Catholic countries (62\% Roman Catholics and 13,4\% atheists in
the census 2011) and the consecrated persons are more numerous and more
appreciated than in the Czech Republic. Nevertheless, the modern history
of these two countries is firmly connected and the convents and
monasteries in both of them are still carrying the consequences of the
long persecution, overworking and forced isolation during the communist
regime. Compared to the previous studies, in view of these facts, we
suppose a higher level of negative health affecting factors and lower
scores of well-being and life satisfaction among the members of the
Czech, possibly Slovak religious communities. We also presume some
slight differences between the Czechs and Slovaks.\\
Therefore, the aim of this study is to explore the possible relations
between the membership in a convent or monastery religious community in
a highly secular post-communist country and both the physical and mental
health of the individuals.

\hypertarget{methods}{%
\section{Methods}\label{methods}}

\hypertarget{measures}{%
\subsection{Measures}\label{measures}}

\hypertarget{health-complains}{%
\subsubsection{Health complains}\label{health-complains}}

The frequency of recent health complains was assessed by the 12-item
measure: headache, stomachache, backache, intestinal problems, feeling
depressed, irritability/bad mood, nervousness, trouble falling asleep,
dizziness, sore throat/cold, heart pounding/chest pain, tingling in
limbs or face. The question was: ``In the past month, how often have you
had the following issues?{}`` Each item was answered on a five-point
scale: never (1), about once or twice (2), approximately once a week
(3), more than once a week (4), every day (5). For analytical purposes,
participants responses were dichotomised. Answers ranging from
1:''never" up to 3: ``approximately once a week'' were recoded as ``Not
many times per week'' and answers ranging from 4: ``More than once a
week'' up to 5: ``every day'' were recoded as ``Many times a week''.

\hypertarget{long-lasting-ilnessess}{%
\subsubsection{Long lasting ilnessess}\label{long-lasting-ilnessess}}

For the occurrence of chronic illnesses, a 26-item measure was used
introduced by the question: ``Do you have a long-term illness or
disability? Please tick all that apply to you.`` The illnesses are
listed in table XY.

\hypertarget{mental-problems-solving}{%
\subsubsection{Mental problems solving}\label{mental-problems-solving}}

The need to find help in case of mental problems was assessed by two
additional questions responded yes or no: ``Are you taking any
psychiatric medication now or have you used it in the past? Have you
ever visited or are you currently visiting a psychologist or
psychotherapist?{}``

\hypertarget{chronotype}{%
\subsubsection{Chronotype}\label{chronotype}}

Daily energy was measured by choosing between two possibilities -- an
early bird who wakes up early but is tired in the evening, or a night
owl who has problem with waking up early but enjoys working in the
evening.

\hypertarget{participants}{%
\subsection{Participants}\label{participants}}

\hypertarget{sample-one}{%
\subsubsection{Sample one}\label{sample-one}}

First sample (\emph{n} = 1800, Age: \emph{M} = 46.41, \emph{SD} = 17.4,
Females: 51.28\%) consisted of participants from Czech nationally
representative sample of the study on health, life experience, attitudes
and lifestyle collected in 2016 \citep{malinakova2020religiosity}. In
this dataset we did not find subjects responding incongruently to the
control items i.e.~feeling the God presence despite being Non-religious
or atheist. Thus, no participant was excluded from a dataset.

\hypertarget{sample-two}{%
\subsubsection{Sample two}\label{sample-two}}

The second sample was collected in April 2020. It was a survey made in
the Czech population during the first Covid 19-lockdown. From the
original dataset (\emph{n} = 1263), we excluded 120 participants who
responded incongruently to 3 repeatedly asked questions and those, who
were speeders i.e.~time spend filling questionnaire was \textless{} 10
min. The three control questions included age (difference \textgreater{}
2 years), weight and height (difference \textgreater{} 2 kilogram and
centimes). Hence, the number of participants was 1143. Based on the
results of outliers screening procedure (see statistical analysis
secion), we also removed subjects, which responded to large number of
questions in the same way (\emph{n} = 2). Therefore, the final number of
participants was 1141 (Age: \emph{M} = 49.2, \emph{SD} = 16.73, Females:
46.45\%).

\hypertarget{sample-three}{%
\subsubsection{Sample three}\label{sample-three}}

The third sample (\emph{n} = 1662) was collected during May 2021 (zde
moc psl dopln další info). After data were collected, we excluded
participants (\emph{n} = 166) reporting incongruent answers and those
who were classified as speeders. The criteria were the same as in the
second sample. This resulted in 1496 (Age: \emph{M} = 50.67, \emph{SD} =
15.79, Females: 44.05\%) participants. No participant with uniform
responses was detected.

\hypertarget{sample-four}{%
\subsubsection{Sample four}\label{sample-four}}

A sample of Catholic order members in the Czech Republic and in the
Slovak Republic was recruited to take part in a survey regarding various
aspects of today´s consecrated life. The respondents were recruited by
contacting the major superiors of all male and female religious
institutes in both countries. After six weeks, the information about the
survey was sent directly into the local communities in order to increase
the number of respondents. The research was done under the auspices of
the Conference of Major Religious Superiors of the Czech and Slovak
Republic. The superiors were asked to spread an online or a
paper-and-pencil questionnaire among the members of their communities
and to support its completing. Data was collected from March to May
2021. This sample initially consisted of 497 participants. In the first
step, we excluded participants (n = 4) who were classified as speeders
i.e.~finished questionnaire typically lasting more than 30 minutes in
\textless{} 10 minutes. After this exclusion, 493 participants remained.
We also removed participants who filled questionnaire multiple times
(\emph{n} = 63) resulting in (n = 430) of subjects . This sample
consisted of 180 Czech participants and 213 Slovak participants. Other
than Czech and Slovak participants were excluded (n = 37) resulting in
393 subjects (Age: M = 47.52, SD = 9.57, Females: 78.88\%). No uniform
pattern of responding was detected in this sample.

\hypertarget{statistical-analysis}{%
\subsection{Statistical analysis}\label{statistical-analysis}}

As suggested by Shapiro-Wilk test and by histograms, normality
assumption was broken in all samples. Thus, non - parametric methods
were used. Homogenity of variances was equal in all samples as indicated
by the Breusch-Pagan test. As the null hypotheses of the MCAR test in
all our surveys was not rejected, we deleted missing values listwise.
Outliers were explored by the Median Absolute Deviation (MAD). Outliers
identified by the MED were consequently screened and if there were signs
of uniform pattern of responding i.e.~answering the number of items in
the same manner, than outlier were removed from the dataset.\\
To explore differences in health status among clerics and non - clerics,
we compared in logistic regression models long lasting illnesses of
clerics to reported long lasting illnesses of participants from
representative sample. In these models, reported long lasting illness
was was as a dependent variable, grouping variable distinguishing
clerics from non - clerics was regressor and covariates were are,
gender, education and length of a life in clerical order.

them to the representative sample diagnoses of ilnesses reported by
participants on the variable

\hypertarget{results}{%
\section{Results}\label{results}}

The table 1 depicts basic socio-demograpthic characteristics of the
study samples.

\newpage

\begin{table}

\caption{\label{tab:socio-demographic table}Socio-demographic table}
\centering
\resizebox{\linewidth}{!}{
\begin{tabular}[t]{lllll}
\toprule
\multicolumn{1}{c}{ } & \multicolumn{1}{c}{Sample 1} & \multicolumn{1}{c}{Sample 2} & \multicolumn{1}{c}{Sample 3} & \multicolumn{1}{c}{Sample 4} \\
\cmidrule(l{3pt}r{3pt}){2-2} \cmidrule(l{3pt}r{3pt}){3-3} \cmidrule(l{3pt}r{3pt}){4-4} \cmidrule(l{3pt}r{3pt}){5-5}{}
Characteristic & N = 1,800 & N = 1,141 & N = 1,496 & N = 393\\
\midrule{}
Gender &  &  &  & \\
\hspace{1em}Female & 923 (51\%) & 530 (50\%) & 659 (44\%) & 310 (79\%)\\
\hspace{1em}Male & 877 (49\%) & 523 (50\%) & 835 (56\%) & 83 (21\%)\\
Family\_status &  &  &  & \\
\hspace{1em}Not in relationship & 439 (24\%) & 267 (25\%) & 201 (13\%) & \\
\addlinespace
\hspace{1em}Married & 929 (52\%) & 461 (44\%) & 714 (48\%) & \\
\hspace{1em}Divorced & 158 (8.8\%) & 201 (19\%) & 252 (17\%) & \\
\hspace{1em}Widow/Widower & 133 (7.4\%) & 73 (6.9\%) & 91 (6.1\%) & \\
\hspace{1em}In relationship & 141 (7.8\%) & 51 (4.8\%) & 236 (16\%) & \\
Education &  &  &  & \\
\addlinespace
\hspace{1em}Basic school & 141 (7.8\%) & 90 (8.7\%) & 91 (6.1\%) & 1 (0.3\%)\\
\hspace{1em}Vocational school or non - maturity high school & 442 (25\%) & 400 (39\%) & 572 (38\%) & 12 (3.1\%)\\
\hspace{1em}High school & 854 (47\%) & 377 (36\%) & 451 (30\%) & 48 (12\%)\\
\hspace{1em}Higher vocational school or University & 363 (20\%) & 169 (16\%) & 380 (25\%) & 330 (84\%)\\
\hspace{1em}other: ošetrovatelský kurz &  &  &  & 1 (0.3\%)\\
\addlinespace
\hspace{1em}other: PhD. &  &  &  & 1 (0.3\%)\\
Economical\_status &  &  &  & \\
\hspace{1em}Without work & 261 (14\%) & 149 (14\%) & 172 (13\%) & \\
\hspace{1em}Pensioner & 430 (24\%) & 325 (31\%) & 420 (32\%) & \\
\hspace{1em}Working & 1,109 (62\%) & 559 (54\%) & 707 (54\%) & \\
\addlinespace
Faith &  &  &  & \\
\hspace{1em}Yes, I am a member of church & 170 (9.4\%) &  & 132 (9.4\%) & \\
\hspace{1em}Yes, but I am not a member of a church & 361 (20\%) &  & 331 (24\%) & \\
\hspace{1em}No & 1,004 (56\%) &  & 680 (48\%) & \\
\hspace{1em}No, I am convinced atheist & 265 (15\%) &  & 262 (19\%) & \\
\bottomrule{}
\end{tabular}}
\end{table}

\[\\[1in]\]

\newpage

\hypertarget{chronotype-differences}{%
\subsection{Chronotype differences}\label{chronotype-differences}}

In the first step of the analysis, we compared clerics with non -
clerics in their self - reported chronotype. Pearson, chi-square test
revealed that there was no difference between these two groups across
the two surveys see Table 2.

\begin{table}[!h]

\caption{\label{tab:contingenty tables}Differences in chronotype across samples (N = 2731)}
\centering
\begin{tabular}[t]{llll}
\toprule
\multicolumn{1}{c}{ } & \multicolumn{2}{c}{Chronotype} & \multicolumn{1}{c}{ } \\
\cmidrule(l{3pt}r{3pt}){2-3}{}
Characteristic & Early bird, N = 1,496 & Night own, N = 1,235 & p-value\\
\midrule{}
Source &  &  & 0.2\\
\hspace{1em}Panel & 498 (33\%) & 453 (37\%) & \\
\hspace{1em}Vacctination & 812 (54\%) & 632 (51\%) & \\
\hspace{1em}Consecrated & 186 (12\%) & 150 (12\%) & \\
\bottomrule{}
\end{tabular}
\end{table}

\hypertarget{chronical-ilness-differences}{%
\subsection{Chronical ilness
differences}\label{chronical-ilness-differences}}

Regression analysis indicated that there is a significnat positive
relationsihp between being cleric and lower probabitly of:

\begin{table}[!h]

\caption{\label{tab:Print regres tab 1}Depicts associations (in Odds rations) between living in clerical life and chronical mental and physical deseases}
\centering
\resizebox{\linewidth}{!}{
\fontsize{7}{9}\selectfont
\begin{threeparttable}
\begin{tabular}[t]{lllll}
\toprule{}
Allergy & Migraine & Pain of unclear origin & Pain in the small pelvis & Depression/Anxiety\\
\midrule{}
1.09 (0.72, 1.60) & 0.75 (0.41, 1.25) & 0.92 (0.41, 1.82) & 1.73 (0.82, 3.29) & 2.18** (1.31, 3.49)\\
1.15 (0.45, 2.84) & 0.71 (0.18, 2.40) & 1.66 (0.28, 7.30) & 2.54 (0.50, 10.9) & 1.09 (0.32, 3.40)\\
Ischemic heart disease & Hypertension & Stroke & Astma & \\
0.33 (0.05, 1.06) & 0.84 (0.54, 1.26) & 1.15 (0.18, 3.98) & 1.04 (0.57, 1.76) & \\
0.06 (0.00, 1.61) & 0.17* (0.04, 0.59) & 0.00 (0.00, 0.52) & 0.54 (0.13, 1.97) & \\
Cancer & Diabetes & Obesity & Arthritis & \\
1.61 (0.55, 3.81) & 0.29** (0.10, 0.65) & 1.52 (0.93, 2.38) & 1.05 (0.52, 1.90) & \\
0.18 (0.01, 2.30) & 0.26 (0.02, 1.99) & 0.67 (0.18, 2.08) & 0.09* (0.01, 0.62) & \\
Back pain & Gastric or duodenal ulcers & Chronic lung disease & Skin diseases eczema & \\
0.97 (0.69, 1.36) & 0.61 (0.15, 1.67) & 0.47 (0.03, 2.27) & 1.11 (0.61, 1.88) & \\
0.81 (0.34, 1.84) & 1.00 (0.05, 9.07) & 0.61 (0.03, 3.62) & 0.58 (0.13, 2.13) & \\
\bottomrule{}
\end{tabular}
\begin{tablenotes}[para]
\item \textit{Note:} 
\item p < 0.05 *, p < 0.01 ** p < 0.001 ***, Adjusted effect was calculated using the following variables as a covariates: Age, Gender Education. Values in brackets indicates 95\% confidence interval
\end{tablenotes}
\end{threeparttable}}
\end{table}

porovnat řeholníky slováky s řeholníky čechy

\hypertarget{subsection-heading-here}{%
\subsection{Subsection Heading Here}\label{subsection-heading-here}}

\hypertarget{subsubsection-heading-here}{%
\subsubsection{Subsubsection Heading
Here}\label{subsubsection-heading-here}}

Bulleted lists look like this:

\hypertarget{discussion}{%
\section{Discussion}\label{discussion}}

\hypertarget{conclusion}{%
\section{Conclusion}\label{conclusion}}

\hypertarget{patents}{%
\section{Patents}\label{patents}}

% %%%%%%%%%%%%%%%%%%%%%%%%%%%%%%%%%%%%%%%%%%
% %% optional
% \supplementary{The following are available online at www.mdpi.com/link, Figure S1: title, Table S1: title, Video S1: title.}
%
% % Only for the journal Methods and Protocols:
% % If you wish to submit a video article, please do so with any other supplementary material.
% % \supplementary{The following are available at www.mdpi.com/link: Figure S1: title, Table S1: title, Video S1: title. A supporting video article is available at doi: link.}

\vspace{6pt}

%%%%%%%%%%%%%%%%%%%%%%%%%%%%%%%%%%%%%%%%%%
\acknowledgments{All sources of funding of the study should be
disclosed. Please clearly indicate grants that you have received in
support of your research work. Clearly state if you received funds for
covering the costs to publish in open access.}

%%%%%%%%%%%%%%%%%%%%%%%%%%%%%%%%%%%%%%%%%%
\authorcontributions{For research articles with several authors, a short
paragraph specifying their individual contributions must be provided.
The following statements should be used ``X.X. and Y.Y. conceive and
designed the experiments; X.X. performed the experiments; X.X. and Y.Y.
analyzed the data; W.W. contributed reagents/materials/analysis tools;
Y.Y. wrote the paper.'\,' Authorship must be limited to those who have
contributed substantially to the work reported.}

%%%%%%%%%%%%%%%%%%%%%%%%%%%%%%%%%%%%%%%%%%
\conflictsofinterest{Declare conflicts of interest or state `The authors
declare no conflict of interest.' Authors must identify and declare any
personal circumstances or interest that may be perceived as
inappropriately influencing the representation or interpretation of
reported research results. Any role of the funding sponsors in the
design of the study; in the collection, analyses or interpretation of
data in the writing of the manuscript, or in the decision to publish the
results must be declared in this section. If there is no role, please
state `The founding sponsors had no role in the design of the study; in
the collection, analyses, or interpretation of data; in the writing of
the manuscript, an in the decision to publish the results'.}

%%%%%%%%%%%%%%%%%%%%%%%%%%%%%%%%%%%%%%%%%%
%% optional
\abbreviations{The following abbreviations are used in this manuscript:\\

\noindent
\begin{tabular}{@{}ll}
MDPI & Multidisciplinary Digital Publishing Institute \\
DOAJ & Directory of open access journals \\
TLA & Three letter acronym \\
LD & linear dichroism \\
MSE & Mean Square Error \\
\end{tabular}}

\input{"appendix.tex"}

%%%%%%%%%%%%%%%%%%%%%%%%%%%%%%%%%%%%%%%%%%
% Citations and References in Supplementary files are permitted provided that they also appear in the reference list here.

%=====================================
% References, variant A: internal bibliography
%=====================================
%\reftitle{References}
%\begin{thebibliography}{999}
% Reference 1
%\bibitem[Author1(year)]{ref-journal}
%Author1, T. The title of the cited article. {\em Journal Abbreviation} {\bf 2008}, {\em 10}, 142--149.
% Reference 2
%\bibitem[Author2(year)]{ref-book}
%Author2, L. The title of the cited contribution. In {\em The Book Title}; Editor1, F., Editor2, A., Eds.; Publishing House: City, Country, 2007; pp. 32--58.
%\end{thebibliography}

% The following MDPI journals use author-date citation: Arts, Econometrics, Economies, Genealogy, Humanities, IJFS, JRFM, Laws, Religions, Risks, Social Sciences. For those journals, please follow the formatting guidelines on http://www.mdpi.com/authors/references
% To cite two works by the same author: \citeauthor{ref-journal-1a} (\citeyear{ref-journal-1a}, \citeyear{ref-journal-1b}). This produces: Whittaker (1967, 1975)
% To cite two works by the same author with specific pages: \citeauthor{ref-journal-3a} (\citeyear{ref-journal-3a}, p. 328; \citeyear{ref-journal-3b}, p.475). This produces: Wong (1999, p. 328; 2000, p. 475)

%=====================================
% References, variant B: external bibliography
%=====================================
\reftitle{References}
\externalbibliography{yes}
\bibliography{mybibfile.bib}

%%%%%%%%%%%%%%%%%%%%%%%%%%%%%%%%%%%%%%%%%%
%% optional
\sampleavailability{Data used for the analysis in this study as well as
the code are publically avalible nad can be found on the Open Scienfe
Network webside (\url{https://osf.io/ad6b3/}).}

%% for journal Sci
%\reviewreports{\\
%Reviewer 1 comments and authors’ response\\
%Reviewer 2 comments and authors’ response\\
%Reviewer 3 comments and authors’ response
%}

%%%%%%%%%%%%%%%%%%%%%%%%%%%%%%%%%%%%%%%%%%
\end{document}
